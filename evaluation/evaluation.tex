\chapter{\label{chap:evaluation}Evaluation}

\section{Code quality}
For analyzing the code quality of the MusicDAO application, we can use code coverage measurements. The code coverage is shown in \ref{tab:code-cov}, measured by Android Studio 4\footnote{\url{https://developer.android.com/studio}}. The majority of uncovered code contain user interface interaction and networking callback logic. Code coverage could have been improved by introducing Android instrumented tests. These are tests that run on an Android device or emulator so that user interaction flows can be tested. However, we chose to not implement this as these type of tests can not be run in our continuous integration environment, and tweaking the CI for support of this would be a non-trivial task.
\begin{table}[]
\begin{tabular}{|l|l|}
\hline
\textbf{Package/class} & \textbf{Line coverage} \\ \hline
MusicService.kt        & 15\% (15/95)           \\ \hline
MusicBaseFragment.kt   & 50\% (1/2)             \\ \hline
catalog                & 24\% (35/142)          \\ \hline
dialog                 & 20\% (24/130)          \\ \hline
ipv8                   & 81\% (57/70)           \\ \hline
net                    & 90\% (27/30)           \\ \hline
player                 & 0\% (0/50)             \\ \hline
playlist               & 9\% (19/203)           \\ \hline
util                   & 60\% (42/70)           \\ \hline
wallet                 & 0\% (0/100)            \\ \hline
\end{tabular}
\caption{Code coverage overview}
\label{tab:code-cov}
\end{table}