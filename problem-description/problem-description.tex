\chapter{\label{chap:related-work}Problem description}
% Centralization of data, with the incentive of making money, naturally leads to a centralization of power. 
The most widely used music streaming services, with the largest music catalogs, run centralized, proprietary and cosed-source software. The companies owning these services have a large amount of power in the music streaming industry because of their scale. The top 5 streaming services have a combined market share of 81\%, so this can be regarded as an oligarchy. Because of their power, they can ask high commission fees or lock artists to one platform. As a result, artists receive low compensation. Furthermore, in closed-source software, the processing and storing of user data is nontransparent. The recommendation and playlist generation algorithms are also a black box for the user. As companies make money from selling data to third parties, their data-gathering methods are expected to become more disruptive for user privacy. At the time of writing, there exists no alternative decentralized and transparent music streaming system without intermediaries.

\textit{How can we design and implement a music streaming service that distributes the power from one authority to its users?}% \subsection{Ownership} ? It may be true that when you purchase a record on iTunes or favourite something on Spotify, you do not legally own a copy of it but only an access to it via their proprietary closed source platform.
\subsection{Intermediaries take a large share}
Artists publishing their content on Big Tech music platforms such as Google Music, Spotify and iTunes receive low compensation, because the intermediaries take a large share. According to Midia Research, the top 5 most popular music streaming services control 81\% market share\cite{midiamarketshare2020}, which can be regarded as a  Specifically, these companies take on average a 25 percent cut for signed records, and a 40 percent cut for unsigned records\footnote{\url{https://www.theguardian.com/technology/2015/apr/03/how-much-musicians-make-spotify-itunes-youtube}}. 
% In addition, these platforms use a "pro rata" model, meaning that the artists are paid per amount of plays per month. This model makes it more difficult for small, independent artists to make a living \todo{add sources}. 
% Content creators on the Internet receive a low compensation for publishing their content on monopolize
% Expand upon: small independent artists. Smaller artists get low revenue, due to the "pro rata" model applied by e.g. Spotify. A user-centric model should be more fair.
\subsection{Monopsony power}
% \section{Centralization of power}
% Centralization of data, with the incentive of making money, naturally leads to a centralization of power
Monopsony power means that a dominant buyer has the power to push prices down with suppliers. In the context of music, this means that artists have little choice over which platform to publish their music on, because of the dominance of one platform. A few major players in the music industry together form an oligopolistic market. Monopsony power in this area can lead to squeezing the producer side. An example of monopsony power is an event that happened in 2014, between Amazon and Hachette. Amazon, having a large market share on e-books, used its commercial muscle to demand a larger cut of the price of Hachette books it sells. This included for all Hachette books ``preventing customers from being able to pre-order titles, reducing the discounts it offered on books and delaying shipment''\cite{theguardian2014amazon}. Along the same lines, Spotify can use its commercial muscle to demand low pays to artists. If some artists are not willing to cooperate, Spotify can deliberately remove their content from its recommendations, such as from the Discover Weekly playlists. Spotify claims that over 10 billion times a month, listeners across both Spotify and Spotify Premium stream a new artist they had never heard before. So their recommendation system is highly influential for its users.
\subsection{Transparency of user data processing}
Big Tech companies obtain personal usage data to be able to improve their service, but also to to sell the data to third parties for a profit. In this process, users must heavily trust the company running the service to handle their data exactly as stated in their privacy policy. In the scope of music streaming services, user data such as browsing activity, and friends and sharing activity is obtained. For instance, Spotify saves personal usage data such as ``search queries [...], streaming history, playlists you create, your library, your browsing history, and your interactions with the Spotify Service, content, other Spotify users.'' and shares this data with advertising parties, stated in their privacy policy\footnote{\url{https://www.spotify.com/us/legal/privacy-policy}}. Google Music ``shares, processes, and maintains information about your usage, access, and playback of Your Music [...], playback activity related to items you preview and buy in the Google Music Store ("Store Usage"); and about the songs you share and listen to in connection with Google Music Social Recommendations [...]'' as described in their privacy policy\footnote{\url{https://music.google.com/about/privacy.html?em_x=22}}. 

Following the lack of privacy comes issues with what companies do with all the user data they gather. Widely known is the use of this data for targeted advertising\cite{jessup2012big} and for selling as a profit\cite{yap2011user}. A risk in this process is that the third parties may use this private information for malicious purposes\cite{yap2011user}. From the perspective of the user, there is no transparency for whether this happens. According to \cite{narayanan2008robust}, privacy may be breached even when a service is willing to protect a user's privacy, because state-of-the-art de-anonymization methods do not fully make users anonymous, depending on the features stored in the database. Centralized software services are subject to a single point-of-failure. In this context this involves the risk of security breaches: if a malicious party gains access to its database, all of the records can be leaked at once which can lead to a large-scale privacy breach. Furthermore, \cite{dworkdifferential} shows that such an event can leak personal data of users who are not part of the original database.
% Integrations with IoT devices and sensors are in development which may lead to more invasion of privacy in the future.
% \subsection{Control of data}
% The GDPR contains the right for individuals to have their data erased from any platform. However, a user taking this action cannot be certain of this happening on request, as access to the companies' database is not available from the outside. Furthermore, the company implements disclosure preferences in the way they see fit, which may not be fine-grained. 
\subsection{Content censoring}
The company running the software is free in how and which content to censor. In addition, their content censoring policy may be changed at any time. Recent examples exist such as the disappearance of Li Zhi\footnote{\url{https://www.independent.co.uk/news/world/asia/tiananmen-square-china-li-zhi-singer-disappears-anniversary-protests-a8940641.html}}, who published songs about democracy and social issues in China. All of China's main streaming sites removed his songs. In 2019, Apple Music removed content from their platform by singer Jacky Cheung, who referenced the tragedies of Tiananmen Square in his songs.
\subsection{Recommendation of content}
The Big Tech music companies recommend content that best fits their business model, which may be contrary to what fits the user best. The companies can promote or dis-promote content by their choosing. For example, on Spotify, brands are able to sponsor playlists. ``A car company might sponsor a popular driving playlist on Spotify''\cite{prey2018nothing}. As the companies run closed-source software, the recommendation engine they use are a black box to the user. They are not obliged to explain the algorithms used for this. Small, independent artists may suffer from labels that invest large amounts of money to have their content promoted.
\subsection{Security and fault tolerance}
As the service and software are proprietary, and the running code is closed-source, there are security risks. Specifically, the cryptography and security mechanisms used internally can not be inspected by people outside of the company. \todo[inline]{expand}
% \subsection{Server costs} ? (This is not really a problem from the perspective of user and/or artist? Unless there are sources that say otherwise)
\subsection{Resiliency}
At any point in time, the company running proprietary software can change, add or remove features. Its users do not necessarily have a vote in this. When a software service is sold to a different owner, the new owner can completely change direction for the service, which makes the service prone to large, possibly unwanted changes. Moreover the company can decide to take down the service in its entirety. For example: In 2017, Pandora discontinued running its service in New Zealand and Australia\footnote{\url{https://www.businessinsider.nl/pandora-shutting-down-services-australia-new-zealand-2017-7?international=true&r=US}}. In this case, users can lose all their data stored on the service.
% All the evidence suggests a description of an oligopolistic market - a few large providers with high market share, interdependence based upon pricing (chance for a kinked demand curve to be drawn, if your awarding body still uses that model!) and only slightly differentiated products.
\subsection{Platform locking}
As an example from YouTube, a company can disallow content creators to publish their work on other platforms, resulting creators to be locked to one platform. \todo[inline]{Find sources of this happening}  
% \subsection{Oligopoly formation}
% Also, a natural consequence: (Unnecessary high) pricing of content. When a monopoly on music content is formed, the Big Tech companies can together decide to up the price 'whenever they feel like it'.
% \subsection{Copyright orthodoxy} ?
% % https://search.proquest.com/openview/2ea25b1bed66750c91cdbc6ea2a5093a/1
% \subsection{Low level of innovation} ?

% \todo[inline]{Software quality measurements: functionality, scalability, maintainability, portability, and, ingeneral, enhancing the usability of the software. See \cite{raghunathan2005open}}

% From Audius Whitepapter:
% We see a number of specific challenges faced by creatorsand listeners today:1.  There is little to no transparency around the originsof creator payouts (e.g.  number of plays, location,original gross payment before fees)2.  Incomplete  rights  ownership  data  often  preventscontent  creators  from  getting  paid;  instead,  earn-ings accumulate in digital service providers (DSPs)and rights societies3.  There are layers of middlemen and significant timedelay involved in payments to creators4.  Publishing rights are complicated and opaque, withno incentives for the industry to make rights datapublic and accurate5.  Remixes,  covers,  and  other  derivative  content  arelargely censored due to rights management issues6.  Licensing issues prevent DSPs and content from be-ing accessible worldwide


% Spotify’s front page “Browse” screen presents a classic illusion of choice, a stream of genre and mood playlists, charts, new releases, and now podcasts and video. It all appears limitless, a function of the platform’s infinite supply, but in reality it is tightly controlled by Spotify’s staff and dictated by the interests of major labels, brands, and other cash-rich businesses who have gamed the system. https://thebaffler.com/salvos/the-problem-with-muzak-pelly
% “The more vanilla the release, the better it works for Spotify. If it’s challenging music? Nah,” 
% We should call this what it is: the automation of selling out.
% “The difference now is that, if you don’t bow down to Spotify, you might as well tell whoever runs the guillotine that’s above your neck to just let her rip,” Saunier says, as the band sits in their van, on tour, en route from Grand Rapids to Detroit. “These streaming services are literally the only option for a music career nowadays.”