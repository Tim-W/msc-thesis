\chapter{\label{chap:related-work}Problem description}
% Centralization of data, with the incentive of making money, naturally leads to a centralization of power. 
The most widely used music streaming services, with the largest music catalogs, run centralized, proprietary and lcosed-source software. The companies owning these services have an ever increasing amount of power in the music streaming industry due to the amount of personal data they have about listeners and artists\todo[inline]{Add other reasons}. Because of their power, they can ask high commission fees or lock artists to one platform. As a result, artists receive low compensation. The processing and storing of user data is nontransparent, as both the database and code are closed for people on the outside. As companies make money from data, their data-gathering methods are expected to become more disruptive for user privacy.

\textit{How can we design a music streaming service as alternative to Big Tech that distributes the power from one authority to its users?}% \subsection{Ownership} ? It may be true that when you purchase a record on iTunes or favourite something on Spotify, you do not legally own a copy of it but only an access to it via their proprietary closed source platform.
\subsection{Intermediaries take a large share}
Artists publishing their content on Big Tech music platforms such as Google Music, Spotify and iTunes receive low compensation, because the intermediaries take a large share. Specifically, these companies take on average a 25 percent cut for signed records, and a 40 percent cut for unsigned records\footnote{\url{https://www.theguardian.com/technology/2015/apr/03/how-much-musicians-make-spotify-itunes-youtube}}. \todo[inline]{expand}
% In addition, these platforms use a "pro rata" model, meaning that the artists are paid per amount of plays per month. This model makes it more difficult for small, independent artists to make a living \todo{add sources}. 
% Content creators on the Internet receive a low compensation for publishing their content on monopolize
% Expand upon: small independent artists. Smaller artists get low revenue, due to the "pro rata" model applied by e.g. Spotify. A user-centric model should be more fair.
\subsection{User privacy}
Big Tech companies obtain personal usage data to be able to improve their service, but also to to sell the data to third parties for a profit. In this process, users must heavily trust the company running the service to handle their data exactly as stated in their privacy policy. In the scope of music streaming services, user data such as browsing activity, and friends and sharing activity is obtained. For instance, Spotify saves personal usage data such as ``search queries [...], streaming history, playlists you create, your library, your browsing history, and your interactions with the Spotify Service, content, other Spotify users.'' and shares this data with advertising parties, stated in their privacy policy\footnote{\url{https://www.spotify.com/us/legal/privacy-policy}}. Google Music ``shares, processes, and maintains information about your usage, access, and playback of Your Music [...], playback activity related to items you preview and buy in the Google Music Store ("Store Usage"); and about the songs you share and listen to in connection with Google Music Social Recommendations [...]'' as described in their privacy policy\footnote{\url{https://music.google.com/about/privacy.html?em_x=22}}. 
\subsection{Data usage}
Following the lack of privacy comes issues with what companies do with all the user data they gather. Widely known is the use of this data for targeted advertising\cite{jessup2012big} and for selling as a profit\cite{yap2011user}. A risk in this process is that the third parties may use this private information for malicious purposes\cite{yap2011user}. From the perspective of the user, there is no transparency for whether this happens. According to \cite{narayanan2008robust}, privacy may be breached even when a service is willing to protect a user's privacy, because state-of-the-art de-anonymization methods do not fully make users anonymous, depending on the features stored in the database. Centralized software services are subject to a single point-of-failure. In this context this involves the risk of security breaches: if a malicious party gains access to its database, all of the records can be leaked at once which can lead to a large-scale privacy breach. Furthermore, \cite{dworkdifferential} shows that such an event can leak personal data of users who are not part of the original database.
% Integrations with IoT devices and sensors are in development which may lead to more invasion of privacy in the future.
\subsection{Control of data}
The GDPR contains the right for individuals to have their data erased from any platform. However, a user taking this action cannot be certain of this happening on request, as access to the companies' database is not available from the outside. Furthermore, the company implements disclosure preferences in the way they see fit, which may not be fine-grained. 
\subsection{Content censoring}
The company running the software is free in how and which content to censor. In addition, their content censoring policy may be changed at any time. Recent examples exist such as the disappearance of Li Zhi\footnote{\url{https://www.independent.co.uk/news/world/asia/tiananmen-square-china-li-zhi-singer-disappears-anniversary-protests-a8940641.html}}, who published songs about democracy and social issues in China. All of China's main streaming sites removed his songs. In 2019, Apple Music removed content from their platform by singer Jacky Cheung, who referenced the tragedies of Tiananmen Square in his songs.
\subsection{Recommendation of content}
The Big Tech music companies recommend content that best fits their business model, which may be contrary to what fits the user best. The companies can promote or dis-promote content by their choosing. For example, on Spotify, brands are able to sponsor playlists. ``A car company might sponsor a popular driving playlist on Spotify''\cite{prey2018nothing}. As the companies run closed-source software, the recommendation engine they use are a black box to the user. They are not obliged to explain the algorithms used for this. Small, independent artists may suffer from labels that invest large amounts of money to have their content promoted.
\subsection{Security and fault tolerance}
As the service and software are proprietary, and the running code is closed-source, there are security risks. Specifically, the cryptography and security mechanisms used internally can not be inspected by people outside of the company. \todo[inline]{expand}
% \subsection{Server costs} ? (This is not really a problem from the perspective of user and/or artist? Unless there are sources that say otherwise)
\subsection{Resiliency}
At any point in time, the company running proprietary software can change, add or remove features. Its users do not necessarily have a vote in this. When a software service is sold to a different owner, the new owner can completely change direction for the service, which makes the service prone to large, possibly unwanted changes. Moreover the company can decide to take down the service in its entirety. For example: In 2017, Pandora discontinued running its service in New Zealand and Australia\footnote{\url{https://www.businessinsider.nl/pandora-shutting-down-services-australia-new-zealand-2017-7?international=true&r=US}}. In this case, users can lose all their data stored on the service. \todo[inline]{Add sources of this happening}
% \section{Centralization of power}
% Centralization of data, with the incentive of making money, naturally leads to a centralization of power
\subsection{Platform locking}
As an example from YouTube, a company can disallow content creators to publish their work on other platforms, resulting creators to be locked to one platform. \todo[inline]{Find sources of this happening}
% \subsection{Oligopoly formation}
% Also, a natural consequence: (Unnecessary high) pricing of content. When a monopoly on music content is formed, the Big Tech companies can together decide to up the price 'whenever they feel like it'.
% \subsection{Copyright orthodoxy} ?
% % https://search.proquest.com/openview/2ea25b1bed66750c91cdbc6ea2a5093a/1
% \subsection{Low level of innovation} ?

% \todo[inline]{Software quality measurements: functionality, scalability, maintainability, portability, and, ingeneral, enhancing the usability of the software. See \cite{raghunathan2005open}}

% From Audius Whitepapter:
% We see a number of specific challenges faced by creatorsand listeners today:1.  There is little to no transparency around the originsof creator payouts (e.g.  number of plays, location,original gross payment before fees)2.  Incomplete  rights  ownership  data  often  preventscontent  creators  from  getting  paid;  instead,  earn-ings accumulate in digital service providers (DSPs)and rights societies3.  There are layers of middlemen and significant timedelay involved in payments to creators4.  Publishing rights are complicated and opaque, withno incentives for the industry to make rights datapublic and accurate5.  Remixes,  covers,  and  other  derivative  content  arelargely censored due to rights management issues6.  Licensing issues prevent DSPs and content from be-ing accessible worldwide